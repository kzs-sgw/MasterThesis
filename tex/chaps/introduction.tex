\chapter{序論}
\label{CBegin}

\section{研究背景}
\label{SBackground}

近年では、コンピュータグラフィックスの表現において、実写並みのリアリズムが要求されるようになってきた。
実物と見間違うほどのCGも珍しくなくなり、聴衆は実物とCGとの差異に敏感になってきている。

\section{本研究の目的}
\label{SObjective}

これまでに紹介したように、技術の進歩にともなって、コンピュータグラフィックスに対する要求は増してきつつある。さらに、近年では実時間での計算

\section{本論文の構成}
\label{SPaper_structure}

\textcolor{red}{***要チェックポイント***}

本論文の構成について述べる。
次の\chapref{CRelatedWork}では、本研究と関連のある技術手法や生物分野で複眼について調査等を行った研究を紹介する。
\chapref{CTechnology}では、周辺技術として本研究の基礎となるシェーダアルゴリズムについて解説を行う。
続いて\chapref{CExperiment}では、過去の研究に基づいて本研究で実際に行った予備実験について説明する。
実験結果を踏まえて\chapref{CMethod}では、新しく提案するシミュレーション手法を述べる。
