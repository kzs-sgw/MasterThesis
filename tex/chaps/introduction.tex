\chapter{序論}
\label{CBegin}

\section{研究背景}
\label{SBackground}

コンピュータグラフィックス(Computer Graphics: CG)は、ハードウェアの発展や高性能コンピュータの普及により、近年ではさらに身近なものになっている。
実物と見間違うほどのCGも珍しくなくなり、聴衆は実物とCGとの差異に敏感になってきている。
CG技術の発展により、コンピュータによって作成された画像に対して実写並みのリアリズムが要求されるようになってきた。
CGの歴史上においても、人間が現実世界で目にするものをそのままコンピュータ上に再現することが当初からの目標とされており、一部では実物とCGとの差異が埋まりつつあると言える。

成熟期に入ったとも考えられるCG技術ではあるが、リアリズムの追求はとどまるところを知らず、さらなる技術向上が目指されている。
また、現実世界にはCGによって十分に再現されていないさまざまな光学的な現象が依然として存在することも事実である。
そのため、多くの研究者たちがCGの表現力をさらに向上させるために研究・開発を行っている。


\subsection{体積や厚みを考慮したコンピュータグラフィックス}
\label{SSVolumerendering}

CGは物体やものがどのように「見えるか」をコンピュータによって作成するために生み出された技術である。
そのため、「ものの見た目」においてもっとも重要である物体表面での光の反射がまず取り扱われる。
しかしながら、実際には臓器などの不透明な物体や煙や炎などといった粒子で満たされた一定の領域や体積を考慮する必要のある対象も存在する。
こうした対象を視覚化(visualize:ビジュアライズ)するためには、物体内部や粒子で満たされた領域内部を通過することによって光の挙動にどのような影響が及ぼされるかについても考慮する必要がある。

たとえば、ボリュームレンダリング(Volume Rendering)という手法では、空間をボクセル(voxel)と呼ばれる六面体に分割し、遮る光の量を計算する。
ボクセルごとの密度も考慮しており、密度の大きなボクセルほど多くの光の量を遮る。
歴史的には、医療の分野で用いられるCT(Computed Tomography)データのビジュアライゼーションを目的とする視覚化手法の研究が重ねられていた。


\subsubsection*{サブサーフェイススキャッタリング}
\label{SSSSSS}



\subsection{生物発想のコンピュータグラフィックス}

生物の発する体色などの色は発色のしくみが解明されていないものもあり、それらは

\subsection{複眼と偽瞳孔}
\label{SSCompoundeyeandpseudopupil}
複雑な内部構造を有し、さらに生物




\section{本研究の目的}
\label{SObjective}

これまでに紹介したように、技術の進歩にともなって、コンピュータグラフィックスに対する要求は増してきつつある。
さらに、近年では実時間での計算

\secref{SSVolumerendering}で述べたように、物体内部を通過する光を考察したレンダリング(rendering)手法が生み出されてはいるものの、より複雑で光の挙動に影響を与えるような内部構造を持つ物体のレンダリングに関しては対象ごとに個別の手法を考案する必要がある。
本研究では、昆虫などの複眼表面に現れる光学現象として\secref{SSCompoundeyeandpseudopupil}で述べた偽瞳孔に着目し、この現象のCGによる表現手法の研究および開発を目的とする。

\section{本論文の構成}
\label{SPaper_structure}

\textcolor{red}{***要チェックポイント***}

本論文の構成について述べる。
次の\chapref{CRelatedWork}では、本研究と関連のある技術手法や生物分野で複眼について調査等を行った研究を紹介する。
\chapref{CKnowledge}では、周辺技術として本研究の基礎となるシェーダアルゴリズムについて解説を行う。
続いて\chapref{CExperiment}では、過去の研究を踏まえて本研究で実際に行った予備実験について説明する。
実験結果を踏まえて\chapref{CMethod}では、本研究で提案するシミュレーション手法を述べる。
