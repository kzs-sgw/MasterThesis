\chapter{関連研究}
\label{CRelatedWork}

%本研究で対象としている複眼の偽瞳孔現象をコンピュータグラフィックスの分野で扱った研究は未だ存在しない。
本研究では、複眼の偽瞳孔現象を対象としている。
偽瞳孔は複眼表面の微細構造によって巨視的に現れる模様であり、偽瞳孔をコンピュータグラフィックスの分野で扱った研究は未だ存在しない。
しかしながら、複眼同様に表面の微細構造を有している材質などのレンダリングを扱った研究は少なからず存在するため、それらを紹介する。

\section{微細構造を扱った研究}
\label{SMicrostructure}

表面微細構造をもつ物質をレンダリングする研究が過去にいくつか行われている。

\subsection{離散的な確率論的マイクロファセットモデル(ざらついた表面のやつ)}

表面のマイクロファセット(microfacet)構造を精細にレンダリングする手法として、Wenzel\cite{}らの行ったハイヒールの表面やクリスマスオーナメントのようにきらめく材質のレンダリングに関する研究がある。
この研究ではサーフェスもしくは観測者の動きに合わせて変化するきらめきのランダムなパターンを表現している。
この手法の対象は鏡のような薄片を含むダイナミックにきらめく表面材質、およびかすかに小さいスケールのきらめきを示す粗い表面材質である。
これらの現象は原則的には高解像度のノーマルマップ(normal map: 法線マップ)によって表現することができる。
しかし、細かな特徴をもつマップは角度のついた照明条件下ではエイリアシング(aliasing)において重大な問題を抱えてしまう。
Wenzelらは、通常連続しているマイクロファセットの分布をサーフェス上の離散的な散乱粒子の分布と置き換えることによって



\subsection{構造色をいくつか}
\subsection{織り目のレンダリングに関するやつ}



\section{本研究の位置づけ}
\label{SPosition}

本研究の位置づけについて述べる。

\textcolor{red}{ほかのやつはたいていランダムにやってもできるけど、本研究の対象である複眼は綺麗に並んでるやつだからそう簡単にはいかないよ}
