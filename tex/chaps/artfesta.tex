\chapter*{応用事例}
\section{かごしまアートフェスタ2014}
\label{SKagoshima}

本研究の応用事例として、2014年12月11日から2014年12月14日の期間に行った「かごしまアートフェスタ2014」への展示の事例を紹介する。

「複眼的宇宙鳥 Bircco (バーコ)と探索する宇宙蜘蛛」と題して、鹿児島市「かごしま県民交流センター」においてアート作品として本手法を応用した展示を行った。
入力機器としてキネクト(Microsoft Kinect センサー)を使用しているため、観客は手を左右にかざすことにより作品中のオブジェクトを回転したり、視点の種類を変更することができる。
本研究では、映画やゲームなどのエンターテインメント分野においてリアリティを向上させるための技術として位置づけていたが、キネクトのようなインタラクティブ技術とを組み合わせることによって、視点の位置によって見え方が変わるという特性自体を活かした新たな研究の方向性を生み出すことができるのではないだろうか。

%% エンタメに貢献できることを考えていたが、展示を行って
%% 視線によって変化する外観と、キネクトのようなインタラクティブ技術とを組み合わせることによって、偽瞳孔の特性を体感的に

%% 映画とかは現実感を上げるための技術だけど、視点の位置によって見え方が変わるという特性自体を


\begin{figure}[htbp]
  \centering
\subfigure[図1]{
\includegraphics*[width=.45\columnwidth]{./img/bircco_green.jpg}
\label{FBirccoGreen}}
\subfigure[図2]{
\includegraphics*[width=.45\columnwidth]{./img/bircco_red.jpg}
\label{FBirccoRed}}\\
\subfigure[図1]{
\includegraphics*[width=.45\columnwidth]{./img/bircco_youjo1.jpg}
\label{FBirccoYoujo1}}
\subfigure[図2]{
\includegraphics*[width=.45\columnwidth]{./img/bircco_youjo2.jpg}
\label{FBirccoYoujo2}}
  \caption{複眼的宇宙鳥 Bircco (バーコ)と探索する宇宙蜘蛛}
  \label{FBircco}
\end{figure}
