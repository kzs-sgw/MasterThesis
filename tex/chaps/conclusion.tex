\chapter{おわりに}
\label{CConclusion}

\section{結論}
\label{SConclusion}

本研究では、昆虫などの複眼に生じる偽瞳孔をCGによってリアルタイムに表現する手法を提案した。
提案手法ではテクスチャと球体レンズによって複眼の光学的モデルを考案し、球体オブジェクトなどに対して手法を適用することで、複眼らしい偽瞳孔の模様を再現することができた。
本研究の提案手法について以下の成果を確認できた。

\begin{itemize}
\item 光の屈折を含む内部構造を考慮したモデルによってcentral pupil, side pupil, second side pupilといった偽瞳孔を再現できる。
\item 個眼レンズを複数通過する光の屈折をリアルタイム処理することができる。
\item 任意のオブジェクト形状に対して複眼シェーダを適用することができる。
\item さまざまなテクスチャを用いることで外観を変えることができ、アート作品への応用として展示を行った。
\end{itemize}

\section{今後の展望}
\label{SFutureWork}

本手法では一般的な六角形の個眼の配置ではなく、格子状の配置を用いている。
そのため、3つもしくは4つのside pupilを表現することはできたが、八木\cite{yagi1951studies}の記述にもあった6つのside pupilを表現することはできていない。
本手法ではテクスチャ利用した球レンズの配置アルゴリズムを用いていたが、三角形ポリゴンのジオメトリ情報を利用した配置アルゴリズムを用いることで改良が可能であると考えられる。
正三角形に近い三角形ポリゴンによって複眼の曲面を構成した場合、各頂点を球レンズの中心座標として用いることができる。
また、この方法を用いれば個眼の配置だけならば実物を比較的正確に再現することがすでに可能となっており\cite{making-of-upside-down}、頂点位置に着目すれば良いため色素細胞のテクスチャとの同期も容易である。
しかしながら、ポリゴン毎ではなく、ひとつの頂点から適用オブジェクト全体を構成するの頂点、面、法線方向などの情報へのアクセスが必要となるため、リアルタイム処理を達成するためには全情報をバッファとしてあらかじめGPUへと送信し保存するなどの工夫が必要だと考えられる。
さらに、\secref{SCompareWithReal}で述べたリング状の偽瞳孔などの、個別の生物に特徴的な偽瞳孔についてはしくみの解明やモデル化のための幾何的解釈が求められる。


