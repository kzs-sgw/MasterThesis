\chapter{本手法の基礎技術}
\label{CBasicmethod}

\section{グラフィックスパイプライン}
\label{SGraphicspipeline}

\section{オブジェクトファイルフォーマット}
\label{SObjfileformat}

モデリングソフト等により前もってオブジェクトファイルを作成し、アプリケーションで読み込みを行う。
さらに、物体形状および法線やテクスチャ座標などの情報を元にアプリケーション内でシェーダへの転送データを生成する。
本手法で必要になるオブジェクトファイルのデータは以下の要素である。

\begin{itemize}
\item 頂点座標値
\item 頂点法線ベクトル
\item テクスチャ座標値
\item 面情報
\end{itemize}

頂点座標値および頂点法線ベクトルはそれぞれ3次元の浮動小数点型、テクスチャ座標値は2次元の浮動小数点型の値となっている。
面情報は、頂点座標値、頂点法線ベクトル、テクスチャ座標値のインデックス番号の組み合わせとなっており、面が表す多角形の頂点の数だけこれらの情報が与えられている。
本手法では三角形ポリゴンのみを対象としているため、この組み合わせが3つずつ続く。

\section{GPU処理}
\label{SGpumethod}

\subsection{シェーダ}
\label{SSShader}
画像処理用演算プロセッサ(GPU:Graphics Processing Unit)で処理を行うのはシェーダとよばれるプログラムであり、このシェーダよって演算処理が実行される。
シェーダ(shader)ではライティング(光源計算)、シェーディング(陰影処理)およびレンダリング(画像ピクセル化)を行う。
本手法では、リアルタイム処理を行うため、プログラマブルパイプライン
