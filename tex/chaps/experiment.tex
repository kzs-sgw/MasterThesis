\chapter{予備実験}
\label{CExperiment}

\section{実験の目的}
\label{SExperimentPurpose}


実際の複眼の構造や形状は複雑であり、コンピュータグラフィックスにおいて実物の形状を作成したり、実物の構造計算に用いることは難しい。
そこで、複眼の性質を表現するために代替となる形状や近似モデルが必要となる。
本研究では、永田\cite{}が行った偽瞳孔の再現実験をもとにビー玉とパンチングメタルを用いて予備実験を行った。

本実験における第一の目的は、 偽瞳孔の発生原理を理解し、物理現象に落としこむことである。
複眼を扱った関連研究のうち、偽瞳孔の性質や現象について生物学上の考察や解説を行うものは存在するものの、幾何学的なしくみについて言及したものは少ない。
そのため、実物を観察することによって物理現象としてのしくみを明らかにし、開発の足がかりとする必要があった。
第二の目的は、実装を行う前に近似手法による偽瞳孔の再現度を確認することである。
永田の実験は複眼の形状を大きく変え、平板と球体として近似している。そのため、コンピュータグラフィックスとして表現するにあたり、適切な手法であることを確認することが望ましい。
そして第三の目的は、レンズや色素細胞に相当するパンチングメタルの穴などの大きさや周期の違いが模様に与える変化を観察することである。

\section{実験方法}
\label{SExperimentMethod}



\section{結果と議論}
\label{SExperimentResult}

