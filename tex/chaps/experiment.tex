\chapter{予備実験}
\label{CExperiment}

\section{実験の目的}
\label{SExperimentPurpose}


実際の複眼の構造や形状は複雑であり、コンピュータグラフィックスにおいて実物の形状を作成したり、実物の構造計算に用いることは難しい。
そこで、複眼の性質を表現するために代替となる形状や近似モデルが必要となる。
本研究では、永田\cite{}が行った偽瞳孔の再現実験をもとにビー玉とパンチングメタルを用いて予備実験を行った。

本実験における第一の目的は、 偽瞳孔の発生原理を理解し、物理現象に落としこむことである。
複眼を扱った関連研究のうち、偽瞳孔の性質や現象について生物学上の考察や解説を行うものは存在するものの、幾何学的なしくみについて言及したものは少ない。
そのため、実物を観察することによって物理現象としてのしくみを明らかにし、開発の足がかりとする必要があった。
第二の目的は、実装を行う前に近似手法による偽瞳孔の再現度を確認することである。
永田の実験は複眼の形状を大きく変え、平板と球体として近似している。そのため、コンピュータグラフィックスとして表現するにあたり、適切な手法であることを確認することが望ましい。
そして第三の目的は、レンズや色素細胞に相当するパンチングメタルの穴などの大きさや周期の違いが模様に与える変化を観察することである。
実際の昆虫などの複眼は各個眼の大きさが決まっており、全体の形状に対して自由に個眼の大きさを変化させることができない。
ゆえに、大きさや周期といったパラメータの変化に対して偽瞳孔の模様が変化する様子を確認するために模型を利用する。

\section{実験方法}
\label{SExperimentMethod}

実験に用いた道具は以下のとおりである\figref{}。

\begin{itemize}
\item ビー玉(透明なもの)
\item パンチングメタル(穴の大きさと周期の違うもの2種類)
\item 黒色の画用紙
\item 木の棒
\end{itemize}

\noindent
まず、画用紙とパンチングメタルをセロハンテープなどで固定する。
これは、偽瞳孔の模様をはっきりとさせるためにパンチングメタルの穴の部分を黒く見せるためである。
次に、木の棒で枠を作りパンチングメタルの上に乗せる。
続いて、木の枠内にできるだけ隙間が開かないようにビー玉を敷き詰める。
このモデルでは、パンチングメタルの穴が複眼の色素細胞に相当し、ビー玉が個眼のレンズに相当する。
実際の複眼では個眼のそれぞれにおいて色素細胞とレンズは対応しており同じ周期で配置されているが、本実験では厳密にパンチングメタルの穴の位置とビー玉の配置を一致させてはいないが、密集した球体レンズによる光の屈折がどのような像を生み出すのかを確認するためには十分である。

\section{結果と議論}
\label{SExperimentResult}

使用したパンチングメタルの大きさは、穴同士の距離約3.0mmで穴の径が約1.0mmのものと穴同士の距離約3.5mmで穴の径が約1.5mmのもの。
ビー玉は径が約12.0mmおよび約8.0mmのものを使用した。
木の棒はそれぞれ幅15.0mm、高さ15.0mm、奥行き200.0mmのもの4本を加工して木の枠を作成した。
また、パンチングメタルの大きさとビー玉の大きさがそれぞれ2種類ずつあるため、これらを組み合わせて4パターンのモデルに対して観察を行った。

まずはじめに、ひとつのビー玉をパンチングメタルの穴の直上に置いて、ビー玉表面に写る像を観察した。
すべてのパターンにおいて黒点(パンチングメタルの穴)が拡大された虚像を確認することができた\figref{}。
さらに、レンズを見る角度を変えてパンチングメタルと平行に近づけていくと、ある角度を堺に倒立像が観測されるようになった\figref{}。
また、球の直下の黒点を像として写した場合よりも、隣接した黒点を写した際に球に占める黒い部分が大きくなるという結果が得られた\figref{}。
以上から、球の直下およびその周辺の黒点に対してビー玉が拡大レンズとして作用していることがわかった。

続いて、木枠を用いてビー玉を密集して配置した\figref{}。
この場合、ビー玉の直下にパンチングメタルの穴があるとは限らないため、各ビー玉ごとに写る像の違いが大きいことに注意する。
ビー玉の大きさを変えても特に大きな違いはなく、各パンチングメタルを拡大した虚像が観測され、生じる像の周期はパンチングメタルの大きさによって変化した。

前述の木枠を用いた方法ではビー玉の周期とパンチングメタルの穴の周期が一致しておらず、球の直下



