\chapter{提案手法}
\label{CMethod}

\secref{SObjective}で述べたように、本研究の目的は複眼表面に観測される光学的現象をリアルタイムレンダリングによって表現することである。
本章では、レンダリング時の計算アルゴリズムについて詳細に解説していく。
まず、\secref{SFramework}で本手法で用いた複眼の近似モデルを提案する。
次に\secref{SObjfile}では、アプリケーションから画像処理用演算プロセッサ(GPU)へと転送するデータと、その生成時に用いた計算アルゴリズムを説明する。
そして\secref{SGpumethod}では、画像処理用演算プロセッサにおいて描画色を求める手法を説明する。

\section{フレームワーク}
\label{SFramework}

\subsection{複眼モデル}
\label{SSModel}

\begin{figure}[h]
  \centering
  \includegraphics[width=3.0in]{./img/TEMP}
  \caption{シェーディングモデル}{\begin{itemize}\item ポリゴンの直下にレンズとして球体を配置。\item 距離を開けてポリゴンと平行にテクスチャ平面を配置。\end{itemize}}
  \label{FModel}
\end{figure}

\chapref{CExperiment}で取り上げた実験をもとに、シェーダアルゴリズムに図のようなモデルを採択した\figref{FModel}。
ポリゴン直下に屈折レンズの役割を果たす球体を配置し、表面を埋め尽くすように多数配置している。
さらに、テクスチャ平面と称してポリゴンと平行な位置にテクスチャ情報を取得するための平面を配置している。
テクスチャ平面は、\secref{}で説明した色素細胞の役割を担っている。
レンズによって屈折された光はテクスチャ平面へ到達し、テクスチャ平面上の情報から色への影響を決定する。

続いて、モデルの利点を説明する。
まず、このモデルは複雑な複眼の構造を球や平面などの単純な幾何立体の集合として扱うことができる。そのため、光の屈折等を計算する際に、きわめて軽量な計算量で済むという利点がある。
次に、\chapref{CExperiment}の実験は永田\cite{}が示しているように偽瞳孔現象の特徴を十分に再現している。
ゆえに、近似モデルとして十分に目的を果たすことが期待できる。


\subsection{アルゴリズムの全体像}
\label{SSAlgorithm}

\begin{figure}[h]
  \centering
  \includegraphics[width=3.0in]{./img/TEMP}
  \caption{アルゴリズムフレームワーク}
  \label{FAlgoframework}
\end{figure}

本手法のアルゴリズムの全体像を説明する。
本手法の構成は大きく分けると、最初に複眼表面を適用する形状データの作成、次にアプリケーションにおいて計算に用いるデータの処理。
そして最後に偽瞳孔による光の減衰量計算および光源、陰影処理等がなされる\figref{FAlgoframework}。
アプリケーションでは通常の手法で作成されたオブジェクトファイル情報を加工し、計算に必要な変数をGPU(シェーダ)へ送る。
シェーダでは屈折などを考慮した物理計算を行い、最終的に画面に描画する色情報を計算する。

\section{テクスチャと球の配列}
\label{STextureandsphere}

テクスチャおよび球は複眼における個眼をモデル化したものである。
そのため、実際の個眼と同様にひとつのテクスチャ単位とひとつの球の位置を合わせて配置する必要がある\figref{FTexturesphereposition}。

\begin{figure}[h]
  \centering
  \includegraphics[width=3.0in]{./img/TEMP}
  \caption{テクスチャと球の位置関係}
  \label{FTexturesphereposition}
\end{figure}

今回用いた手法では、格子状に球を配置させてテクスチャとの位置合わせをおこなう\figref{FGrid}。
具体的には、テクスチャとして用いている正方形画像の各頂点と格子の各点を対応させ、テクスチャ座標軸方向と格子の平行線を一致させる。
正規化されたテクスチャをタイル状に繰り返し適用し、格子の中心と面の法線方向から見た球の中心座標を一致させる。すなわち、法線方向の奥行きは一致しておらず間隙がある。
球の半径を$r$とすると、ポリゴンとテクスチャ平面との距離は$2r$となる\figref{}。

正確には、色素細胞として用いるテクスチャ座標値と球の配列に利用するテクスチャ座標値は独立して扱うことができるため、必ずしも同一のものを利用する必要はない。
ここでは、両者を同一化しても計算上問題が無いためそのまま利用する。

\begin{figure}[h]
  \centering
  \includegraphics[width=3.0in]{./img/TEMP}
  \caption{テクスチャおよび球の格子状配置}
  \label{FGrid}
\end{figure}

\section{計算用データ}
\label{SObjfile}

読み込んだオブジェクトファイルデータから計算用のベクトルデータを生成する。
この処理はアプリケーション内で一度だけ行われ、プログラマブルシェーダへ転送されたのち保持される。


頂点情報としてにシェーダへ転送されるデータは、頂点座標値、頂点法線ベクトル、テクスチャ座標値、そして後述する接ベクトル情報である。
これらのうち頂点座標値および頂点法線ベクトルは更新されず、オブジェクトファイルから読み取った値を面情報にしたがって順次バーテックスバッファオブジェクトの形で配列情報として転送される。
テクスチャ座標値はユーザ指定の浮動小数点型の値であるテクスチャ解像度$R_t$を以下のように乗算し転送される。

\begin{equation}
{\bm T_{vert}} = R_{t}{\bm T_{obj}}
\end{equation}

\noindent
ここで、$\bm{T}_{obj}$は作成した形状データから読み込んだテクスチャ座標値、$\bm{T}_{vert}$は頂点シェーダ(\secref{SSShader})へ転送されるテクスチャ座標値である。

本手法ではテクスチャ座標値をもとに屈折レンズ(\secref{SSModel})の配置を決定しているため、$R_t$を変更することで複眼表面のレンズの配置すなわち表面構造の細かさを任意に変更することができる。

これらの頂点情報の他にシェーダへ与えられる定数などの情報はプログラマブルシェーダ内の変数として適宜転送される。

\subsection{テクスチャ座標軸方向3次元単位ベクトル}
\label{SSUnitvec}

通常、2次元空間上のデータであるテクスチャは、テクスチャ座標値と空間上の点との対応づけにより3次元空間上に描写される。
すなわち、3次元空間における面は2次元の座標空間として考えることができる\figref{}。
そこで、3次元空間内の情報から2次元情報であるテクスチャ座標値を算出するためには、異なる次元同士を橋渡しする変数が必要になる。

本手法では、オブジェクト上の各位置におけるテクスチャ座標系の単位ベクトルを3次元ベクトルとして表すことによって、異なる次元の情報を結びつけている。
本項では、2次元ベクトルであるテクスチャ座標空間の単位ベクトルを、3次元空間上の3次元ベクトルとして表す方法について述べる。
また、3次元空間上のテクスチャ座標軸方向単位ベクトルを用いると、ポリゴン上の任意の点においてテクスチャ座標値を逆算できるようになる。

ポリゴン上のある位置におけるテクスチャ座標空間の軸方向の単位ベクトルを3次元ベクトル$\bm{U}_p, \bm{V}_p$として表すと、ポリゴン上の任意点$\bm{P}$は$\bm{U}_p, \bm{V}_p$を利用して以下のように表現することができる。

\begin{equation}
\bm{P} = \bm{P}_c + a\bm{U}_p + b\bm{V}_p
\label{EPuv}
\end{equation}

\noindent
ここで、$\bm{P}_c$は既知の点$\bm{P}_e$およびそのテクスチャ座標値$(a_e, b_e)$によって

\begin{equation}
\bm{P}_c = \bm{P}_e - a_e\bm{U}_p - b_e\bm{V}_p
\label{EPc}
\end{equation}

\noindent
\equref{EPc}のように表される。また、\equref{EPuv}の$a$および$b$は点$\bm{P}$におけるテクスチャ座標値$(a, b)$を表している。
すなわち、3次元空間上のテクスチャ座標軸方向単位ベクトル$\bm{U}_p, \bm{V}_p$が既知であれば、\equref{EPuv}の係数を利用してポリゴン上の任意の点におけるそのテクスチャ座標値$(a, b)$を逆算することが可能になる。
本手法では、シェーダでの処理に3次元ベクトル$\bm{P}_c$, $\bm{U}_p$および$\bm{V}_p$が必要となるため、これらの値を作成しシェーダへ転送する必要がある。

\subsubsection*{3次元単位ベクトル計算手順}

テクスチャ座標軸方向3次元単位ベクトル$\bm{U}_p$および$\bm{V}_p$は以下の手順で求める。
$\bm{U}_p$および$\bm{V}_p$はポリゴン毎に変化するベクトル変数であり、ポリゴンを構成する各頂点の頂点情報としてバッファに格納される。
まず、本手法で対象としている三角形ポリゴンの各頂点の頂点座標値を$\bm{P}_0,\bm{P}_1,\bm{P}_2$とし、テクスチャ座標値を$\bm{T}_0,\bm{T}_1,\bm{T}_2$とする。
ここでは、頂点座標値が3次元ベクトルであるのに対してテクスチャ座標値が2次元ベクトルであることに留意し、ポリゴン上で両者の対応関係を明確にしていく。

三角形ポリゴンは3次元空間上における平面を一意に表すことができるため、この平面に対応するベクトルを$\bm{P}_0,\bm{P}_1,\bm{P}_2$および$\bm{T}_0,\bm{T}_1,\bm{T}_2$から求めていく。

\begin{figure}[h]
  \centering
  \includegraphics[width=3.0in]{./img/TEMP}
  \caption{ポリゴン上における頂点座標値およびテクスチャ座標値の関係}
  \label{FVertexandtexture}
\end{figure}


頂点座標値から相対ベクトル$\bm{P}_{10}$,$\bm{P}_{20}$を以下のように定義する。

\begin{equation}
\bm{P}_{10} = \bm{P}_1 - \bm{P}_0
\label{EP10}
\end{equation}

\begin{equation}
\bm{P}_{20} = \bm{P}_2 - \bm{P}_0
\label{EP20}
\end{equation}


同様に、テクスチャ座標値から相対ベクトル$\bm{T}_{10}$,$\bm{T}_{20}$を以下のように定義する。

\begin{equation}
\bm{T}_{10} = \bm{T}_1 - \bm{T}_0
\label{ET10}
\end{equation}

\begin{equation}
\bm{T}_{20} = \bm{T}_2 - \bm{T}_0
\label{ET20}
\end{equation}

これらの相対ベクトルはポリゴンのエッジに相当し、それぞれ平面上の変位を表すベクトルとなっている\figref{FVertexandtexture}。

さて、テクスチャ座標軸方向3次元単位ベクトル$\bm{U}_p$および$\bm{V}_p$は$i,j,k,l$を適当な係数として以下のように表すことができる。

\begin{equation}
\bm{U}_p = i\bm{P}_{10} + j\bm{P}_{20}
\label{EUp}
\end{equation}

\begin{equation}
\bm{V}_p = k\bm{P}_{10} + l\bm{P}_{20}
\label{EVp}
\end{equation}

そして、係数$i,j,k,l$は$\bm{T}_{10}$および$\bm{T}_{20}$から導くことができる。


\begin{figure}[h]
  \centering
  \includegraphics[width=3.0in]{./img/TEMP}
  \caption{平面上の変位ベクトル}
  \label{FAtoB}
\end{figure}

続いて、ポリゴン平面上の任意の位置にある点$A$および$B$を考える\figref{FAtoB}。
点$A$の頂点座標値を$\bm{P}_A$そして点$B$の頂点座標値を$\bm{P}_B$とすると、点$A$および点$B$の3次元空間上における位置の変位は、$c,d$を適当な係数として以下のように表すことができる。

\begin{eqnarray}
\bm{P}_{AB} &=& \bm{P}_A - \bm{P}_B\nonumber\\
           &=& c\bm{P}_{10} +  d\bm{P}_{20}  
\label{EPab}
\end{eqnarray}

さらに、$\bm{P}_{10}$と$\bm{T}_{10}$および、$\bm{P}_{20}$と$\bm{T}_{20}$がポリゴン上で対応関係にあることから、点$A$のテクスチャ座標値を$\bm{T}_A$そして点$B$のテクスチャ座標値を$\bm{T}_B$とすると以下の式が成り立つ。

\begin{eqnarray}
\bm{T}_{AB} &=& \bm{T}_A - \bm{T}_B\nonumber\\
           &=& c\bm{T}_{10} +  d\bm{T}_{20}  
\label{ETab}
\end{eqnarray}

ここで、$i,j,k,l$を用いて

\begin{equation}
i\bm{T}_{10} + j\bm{T}_{20} = 
\begin{pmatrix}
1\\
0
\end{pmatrix}
\label{EUnit2u}
\end{equation}

\begin{equation}
k\bm{T}_{10} + l\bm{T}_{20} = 
\begin{pmatrix}
0\\
1
\end{pmatrix}
\label{EUnit2v}
\end{equation}

\noindent
とし、テクスチャ座標空間をUV座標で表すと、2次元空間内において\equref{EUnit2u}はU軸単位ベクトル、\equref{EUnit2v}はV軸単位ベクトルを表すことになる。
さらに、\equref{EPab}および\equref{ETab}の対応関係から\equref{EUp}および\equref{EVp}を導くことができる。

2次正方行列$\bm{A} = (\bm{T}_{10}, \bm{T}_{20})$とすると\equref{EUnit2u}と\equref{EUnit2v}から

\begin{equation}
\bm{A} 
\begin{pmatrix}
i &k\\
j &l
\end{pmatrix}
=
\begin{pmatrix}
1 &0\\
0 &1
\end{pmatrix}
\label{EAx=I}
\end{equation}

\noindent
\equref{EAx=I}が成り立ち、これを変形すると以下のようになる。

\begin{eqnarray}
\label{EIjkl}
\begin{pmatrix}
i &k\\
j &l
\end{pmatrix}
&=& \bm{A}^{-1}
\begin{pmatrix}
1 &0\\
0 &1
\end{pmatrix}\nonumber\\
\nonumber\\
&=& \bm{A}^{-1}
\end{eqnarray}

\noindent
以上から$i,j,k,l$を求めることができる。

さらに、\equref{EPc}において$\bm{P}_e = \bm{P}_0$, $(a_e, b_e) = \bm{T}_0$として$\bm{P}_c$を作成し、$\bm{U}_p$, $\bm{V}_p$と合わせて利用する。

\section{入射球推定}
\label{SFirstsphere}

ポリゴン内部に仮想的に配置した球のうち、どの球と視線が交わるかを求める。
フラグメントシェーダでは球と視線との交点を直接与えられず、ポリゴンと視線ベクトル$\bm{V}$との交点のみが与えられる。
すなわち、ポリゴン上の同じ点に視線が到達したとしても同一の球に入射するとは限らない。
そのため、視線が入射する球を正確に推定する必要がある。

まず、視線ベクトルとポリゴンとの交点を$\bm{P}$とし、その点のテクスチャ座標を$\bm{T} = (a, b)$とすると\secref{SSUnitvec}の$\bm{P}_c$を用いて

\begin{equation}
\label{EP}
\bm{P} = \bm{P}_c + a\bm{U}_p + b\bm{V}_p
\end{equation}

\noindent
と書ける。ここで、$a, b$をそれぞれの値が正のとき

\begin{eqnarray}
\label{Eadashplus}
a' = \lfloor a \rfloor + 0.5\\
\nonumber\\
\label{Ebdashplus}
b' = \lfloor b \rfloor + 0.5
\end{eqnarray}

\noindent
負のとき

\begin{eqnarray}
\label{Eadashminus}
a' = \lceil a \rceil - 0.5\\
\nonumber\\
\label{Ebdashminus}
b' = \lceil b \rceil - 0.5
\end{eqnarray}

\noindent
以上のようにすると、テクスチャの格子の中心を示す$\bm{P}'$は次式のようになり

\begin{equation}
\bm{P}' = \bm{P}_c + a'\bm{U}_p + b'\bm{V}_p 
\label{EPdash}
\end{equation}

\noindent
さらに球の半径$r$および$\bm{P}$における法線ベクトル$\bm{N}$を用いて

\begin{equation}
\bm{S} = \bm{P}' - r\bm{N} 
\label{ESpherepos}
\end{equation}

\noindent
とすると$\bm{S}$は近傍球の中心座標の推定値となる。
推定した球と視線ベクトル$\bm{V}$との交差判定を行い、交差する場合には次の屈折計算へ進む。

しかし、この操作を一度のみ行うだけでは不十分であり正確な近傍球を推定できておらず不具合を生じてしまう。
間違った近傍球推定を行ってしまうと\figref{}のように、表示されるべき球の一部が消失してしまう。
これを防止するために仮想的なスライス平面を作成し、オブジェクト内部方向へ段階的に移動させる\figref{}。
球の半径$r$よりも小さい値のオフセット距離を取り、法線の逆方向へポリゴンと平行に平面を順次作成していく。

視線ベクトル$\bm{V}$を作成した仮想平面まで伸ばし両者の交点位置を求め、再度近傍球の推定を行う。
球の推定にはこの交点位置におけるテクスチャ座標値が必要であるが、3次元空間の点からテクスチャ座標値を逆計算するためには点がポリゴンの平面上に位置している必要がある。
そのため、交点をポリゴンの平面上に投影した位置におけるテクスチャ座標値を再計算する。
交点を$\bm{P}_s$とすると、これをポリゴンの平面上に戻すためには$h$を仮想平面とポリゴンとの距離として

\begin{equation}
\bm{P}_s' = \bm{P}_s + h\bm{N}
\label{EPsdash}
\end{equation}

\noindent
とすればよい。$\bm{P}_s$はポリゴン平面上の点なので

\begin{equation}
\bm{P}_s' = \bm{P}_c + a_s\bm{U}_p + b_s\bm{V}_p 
\label{EPsdash2}
\end{equation}

\noindent
以上の式が成り立つ。未知数$(a_s, b_s)$は最小二乗法を用いて計算を行うと以下のように求めることができる。

\begin{equation}
\bm{P}_{s'c} = \bm{P}_s' - \bm{P}_c\\
\label{EPsdashc}
\end{equation}

\begin{eqnarray}
\label{Eas}
a_s &=& \frac{(\bm{V}_p \cdot \bm{V}_p)(\bm{P}_{s'c} \cdot \bm{U}_p) - (\bm{V}_p \cdot \bm{U}_p)(\bm{P}_{s'c} \cdot \bm{V}_p)}
{(\bm{U}_p \cdot \bm{U}_p)(\bm{V}_p \cdot \bm{V}_p) - (\bm{U}_p \cdot \bm{V}_p)(\bm{V}_p \cdot \bm{U}_p)}\\
\nonumber\\
\nonumber\\
\label{Ebs}
b_s &=& \frac{(\bm{U}_p \cdot \bm{U}_p)(\bm{P}_{s'c} \cdot \bm{V}_p) - (\bm{U}_p \cdot \bm{V}_p)(\bm{P}_{s'c} \cdot \bm{U}_p)}
{(\bm{U}_p \cdot \bm{U}_p)(\bm{V}_p \cdot \bm{V}_p) - (\bm{U}_p \cdot \bm{V}_p)(\bm{V}_p \cdot \bm{U}_p)}
\end{eqnarray}\\
\indent
求めた$a_s, b_s$をもとに、\equref{Eadashplus}から\equref{ESpherepos}までの操作を行い、再度交差判定を行う。
最初に交差した球を第一の入射球として扱い、球の推定探索を終了する。
交差していなければこの仮想平面による処理を仮想平面とポリゴンとの距離が$r$になるまで行い、交差する球があれば\secref{SRefraction}の処理へ進み、最後まで交差する球がなければ次の\secref{STexturerecalculation}の処理へ進む。

\section{屈折計算処理}
\label{SRefraction}




\section{テクスチャ座標再計算}
\label{STexturerecalculation}

\section{光源および陰影処理}
\label{SPhongandshade}
