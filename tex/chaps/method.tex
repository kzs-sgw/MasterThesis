\chapter{提案手法}
\label{CMethod}

\secref{SObjective}で述べたように、本研究の目的は複眼表面に観測される光学的現象をリアルタイムレンダリングによって表現することである。
本章では、レンダリング時の計算アルゴリズムについて詳細に解説していく。
まず、\secref{SFramework}で本手法で用いた複眼の近似モデルを提案する。
次に\secref{SObjfile}では、アプリケーションから画像処理用演算プロセッサ(GPU)へと転送するデータと、その生成時に用いた計算アルゴリズムを説明する。
そして\secref{SGpumethod}では、画像処理用演算プロセッサにおいて描画色を求める手法を説明する。

\section{フレームワーク}
\label{SFramework}

\subsection{複眼モデル}
\label{SSModel}

\begin{figure}[h]
  \centering
  \includegraphics[width=3.0in]{./img/TEMP}
  \caption{シェーディングモデル}{\begin{itemize}\item ポリゴンの直下にレンズとして球体を配置。\item 距離を開けてポリゴンと平行にテクスチャ平面を配置。\end{itemize}}
  \label{FModel}
\end{figure}

\chapref{CExperiment}で取り上げた実験をもとに、シェーダアルゴリズムに図のようなモデルを採択した\figref{FModel}。
ポリゴン直下に屈折レンズの役割を果たす球体を配置し、表面を埋め尽くすように多数配置している。
さらに、テクスチャ平面と称してポリゴンと平行な位置にテクスチャ情報を取得するための平面を配置している。
テクスチャ平面は、\secref{}で説明した色素細胞の役割を担っている。
レンズによって屈折された光はテクスチャ平面へ到達し、テクスチャ平面上の情報から色への影響を決定する。

続いて、モデルの利点を説明する。
まず、このモデルは複雑な複眼の構造を球や平面などの単純な幾何立体の集合として扱うことができる。そのため、光の屈折等を計算する際に、きわめて軽量な計算量で済むという利点がある。
次に、\chapref{CExperiment}の実験は永田\cite{}が示しているように偽瞳孔現象の特徴を十分に再現している。
ゆえに、近似モデルとして十分に目的を果たすことが期待できる。


\subsection{アルゴリズムの全体像}
\label{SSAlgorithm}

\begin{figure}[h]
  \centering
  \includegraphics[width=3.0in]{./img/TEMP}
  \caption{アルゴリズムフレームワーク}
  \label{FAlgoframework}
\end{figure}

本手法のアルゴリズムの全体像を説明する。
本手法の構成は大きく分けると、最初にオブジェクトファイルの読み込み、次ににアプリケーションによる処理およびGPUへの転送。
そして最後にGPUによる処理がなされる\figref{FAlgoframework}。
アプリケーションでは通常の手法で作成されたオブジェクトファイル情報を加工し、計算に必要な変数をGPUへ送る。
GPUでは屈折などを考慮した物理計算を行い、最終的に画面に描画する色情報を計算する。


\section{オブジェクトファイルフォーマット}
\label{SObjfileformat}

モデリングソフト等により前もってオブジェクトファイルを作成し、アプリケーションで読み込みを行う。
さらに、物体形状および法線やテクスチャ座標などの情報を元にアプリケーション内でGPUへの転送データを生成する。
本手法で必要になるオブジェクトファイルのデータは以下の要素である。

\begin{itemize}
\item 頂点座標値
\item 頂点法線ベクトル
\item テクスチャ座標値
\item 面情報
\end{itemize}

頂点座標値および頂点法線ベクトルはそれぞれ3次元の浮動小数点型、テクスチャ座標値は2次元の浮動小数点型の値となっている。
面情報は、頂点座標値、頂点法線ベクトル、テクスチャ座標値のインデックス番号の組み合わせとなっており、面が表す多角形の頂点の数だけこれらの情報が与えられている。
本手法では三角形ポリゴンのみを対象としているため、この組み合わせが3つずつ続く。

\section{転送データ}
\label{SObjfile}

読み込んだオブジェクトファイルデータから転送データを生成する。
この処理はアプリケーション内で一度だけ行われ、GPUへ転送されたのち保持される。


頂点情報としてにGPUへ転送されるデータは、頂点座標値、頂点法線ベクトル、テクスチャ座標値、そして後述する接ベクトル情報である。
これらのうち頂点座標値および頂点法線ベクトルは更新されず、オブジェクトファイルから読み取った値を面情報にしたがって順次バーテックスバッファオブジェクトの形で配列情報として転送される。
テクスチャ座標値はユーザ指定の浮動小数点型の値であるテクスチャ解像度$R_{t}$を以下のように乗算し転送される。

\begin{equation}
{\bm U_{vert}} = R_{t}{\bm U_{obj}}
\end{equation}

\noindent
ここで、$\bm{U}_{obj}$はオブジェクトファイルから読み込んだテクスチャ座標値、$\bm{U}_{vert}$はバーテックスシェーダ(\secref{})へ転送されるテクスチャ座標値である。

本手法ではテクスチャ座標値をもとに屈折レンズ(\secref{SSModel})の配置を決定しているため、$R_{t}$を変更することで複眼表面のレンズの配置すなわち表面構造の細かさを任意に変更することができる。

これらの頂点情報の他にGPUへ与えられる定数などの情報はuniform変数として適宜転送される。

\subsection{接ベクトル作成}
\label{SSTangent}

3次元空間内の情報から2次元情報であるテクスチャ座標値を算出するためには、異なる次元同士を橋渡しする変数が必要になる。
本手法では、オブジェクト上の各位置におけるテクスチャ座標系の単位ベクトルを3次元ベクトルとして表すことによって、異なる次元の情報を結びつけている。

本項では、2次元ベクトルであるテクスチャ座標空間の単位ベクトルを、3次元空間上の3次元ベクトルとして表す方法について述べる。
オブジェクト上のある位置$\bm{P}$におけるテクスチャ座標空間の軸方向の単位ベクトルを3次元ベクトル$\bm{U}_{p}, \bm{V}_{p}$として表すと、3次元空間上の$\bm{P}$および$\bm{U}_{p}, \bm{V}_{p}$によって表現される平面上において以下の式が成り立つ。

\begin{equation}
\bm{P} = \bm{P_0} + a\bm{U}_{p} + b\bm{V}_{p}
\end{equation}

通常、2次元空間上のデータであるテクスチャは、テクスチャ座標と空間上の点との対応づけにより3次元空間上に描写される。
すなわち、3次元空間における面は2次元の座標空間として考えるられる\figref{}。





\section{GPU処理}
\label{SGpumethod}

\subsection{新しいサブsekusyon}
\label{SS}
