\addcontentsline{toc}{chapter}{謝辞}
\chapter*{謝辞}

本論文は多くの方々のご協力により書き上げることが出来ました。
この場を借りて皆様に深く感謝の言葉を申し上げます。

指導教員である東京大学大学院情報学環の河口洋一郎先生には、学部4年生の卒業論文から研究室の学生として大変お世話になりました。
思えば、本論文の研究対象として複眼を選んだのは、先生の昆虫や水生生物への熱い関心からヒントを頂いたからにほかなりません。
私が途中で研究を諦めそうになった時も、先生は真剣に激励してくださいました。
研究室では常に周りの人を気にかけてくださり、不調の時には心配の言葉をかけていただくなど、学生思いなところにはとても感銘を受けました。

私の研究室への配属当時、助教であった米海鵬先生は研究を進める上でのアドバイスをたくさんしてくださり、とても感謝しております。
ご自身の専門とは異なる私の研究に対しても積極的に意見をしていただいたことで、より明確でわかりやすい説明ができるように努めようと思うきっかけになりました。
寂しいことに本年度からはご帰国なさってしまいましたが、母国にてさらにご活躍されることを祈っております。

米先生と入れ替わりで助教として河口研究室に来られた岩澤駿先生にはこの1年という短い期間の中でも数えきれないほどお世話になりました。
修士論文の執筆中も岩澤先生が手作りの差し入れを持ってきてくださったり、進んで快く添削をしてくださったりしたおかげで書き上げることができました。
また、私生活でも私たち学生と同じ目線で接し、博識さが垣間見られる面白い雑談をなさるので毎日退屈しない研究室生活を送ることができました。

職員の上田あいさん、小菅瑞恵さんは研究室の数少ない女性陣としていろんな相談に乗っていただいたり、話し相手になっていただいたりと大変お世話になりました。
河口研究室で初めて、上田さんより年下の「うえだ係」として任命されてからはや2年が経ちましたが、名残惜しくもそろそろ後任に引き継ぎをしなくてはならないようです。
今までありがとうございました。
小菅さんは私のくだらない雑談にも真剣に耳を傾けてくださるのでついつい調子に乗って話しすぎてしまうこともありました。
思い返すとお二方が研究生活の潤滑剤となっていたと言っても過言ではありません。

当時博士課程だった川喜田千晶さんには学部時代からお世話になりました。
研究室のイベント事のたびにテキパキと働くお姿は今でも印象に残っています。

2学年先輩の小倉愛未さん、久住陽介さん、佐藤宏紀さん、團野慎太郎さん、松尾真由さんはお会いした当初から親しくしていただきました。
ご在籍中の時だけでなくOB, OGになられてからもさまざまなイベントを企画され、とても刺激的な学生生活を送ることができました。

1学年先輩の耿金洋さん、新川昌典さん、福嶋昭彦さん、藤澤慶さんには、特に研究面においてお世話になりました。
私たちの就職活動も積極的に助けてくださるなど親身になって接していただいたことにも感謝しております。
また、博士課程に進学された福嶋さんには本年度中までずっと面倒を見ていただき、一番長く付き合わせていただいた先輩として親近感を覚えています。

同学年の磯部俊行くん、金丸修也くんは学部時代も含めると多くの時間、苦楽を共にしてきました。
磯部くんの効率的で無駄のない考え方や論理的な正しさにはいつも驚嘆しています。
私の苦手を補うように支えてくれ、もはや半身として機能していると言えるでしょう。
どうにか吸収合体してパワーアップできないものかと方策を練っています。
大学院から配属になった金丸くんは同学年の中では1年短い研究期間でしたが、いつの間にか着実に研究を遂行していく様子はまるで忍者のようでした。
私も将来修行して金丸くんのような隠密行動術を身につけたいと思います。

1学年後輩の栗本拓紀くん、坂東大毅くんにもお世話になりました。
ふたりともとても熱心でどんどん知識や技術を吸収していくので、先輩として追いつかれないように必死になることで私のモチベーションにもつながりました。
とくに坂東くんはプライベートでもよく遊びに付き合ってくれ、高いコミュニケーション能力を遺憾なく発揮し、とても頼もしかったです。

2学年後輩の小山暁久くん、下村礼介くんは短い期間でしたが少しずつ仲良くなれたと思っています。
ふたりとも河口研究室以外へ進学するのは残念ですが、きっと素晴らしい成果を残してくれると期待しています。

